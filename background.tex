\section{Background}
\label{sec:background}

Microservice architecture is one of the modern software development techniques that arranges an application as a collection of loosely coupled services. Unlike traditional service-oriented architectures (SOA) that rely strongly on products such as enterprise service buses or other, microservices rely only on lightweight technologies such as REST and HTTP web services protocols \cite{Jamshidi:Pahl:Mendonca:Lewis:Tilkov:2018}.

Benefiting from the fast development of cloud computing technique, microservices are the latest trend in software service design, development, and delivery \cite{Zimmermann:2017}. 

Migrating legacy applications from monolithic architecture to microservice architecture has excellent benefits. It creates modularity that makes the application easier to understand, develop, and maintain. It also provides resilience to the system against design erosion \cite{Chen:2018}. In addition, the microservice architecture allows individual components to scale independently regardless of the other components within the system. 

However, decomposing from monoliths to microservices can be very challenging, especially when the scale becomes large. Generally, human efforts are required for a three-stage process: investigating the current behaviour of the monoliths, re-designing the microservices, and actual implementation.

This paper compares the current techniques that can be used for assisting decomposing monolithic applications to microservices. Techniques that are discussed are mainly focusing on the first two stages of the migration.




